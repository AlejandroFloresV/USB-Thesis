\chapter*{Introducción}
\label{intro}
\lhead{\emph{Introducción}}
\addcontentsline{toc}{chapter}{Introducción}

Durante las últimas décadas, los avances tecnológicos hay llevado a un aumento significativo en la cantidad de información generada por la actividad humana. Nuevos campos de estudio han emergido con la finalidad analizar esta enorme cantidad de datos. Bajo el contexto del campo de \emph{Descubrimiento de Conocimiento en Bases de Datos} (KDD por sus siglas en inglés), los procesos de \emph{Minería de Datos} (MD) buscan patrones en los conjuntos de datos con el objetivo de construir modelos de representación que permitan (entre otras cosas) clasificar datos desconocidos. Sin embargo, debido a que estos datos se encuentran agrupados en muestras de un evento particular (o instancias), surge el problema de aparición de instancias redundantes, con datos inconsistentes o ruidosos, etc. En este sentido, la reducción de los datos juega un rol fundamental en la aplicación efectiva de técnicas de MD.

El problema de \emph{Selección de Instancias} (SI) busca escoger un subconjunto de las instancias originales, con la finalidad de reducir la cantidad de datos almacenados sin perjudicar (o incluso mejorando) la capacidad de representación original. Puede verse como un \emph{problema de optimización combinatoria} en el que cada instancia puede pertenecer o no al subconjunto seleccionado; esto implica un espacio de posibles soluciones de tamaño $2^n$, siendo $n$ el número de instancias iniciales. Este problema ha sido estudiado ampliamente por númerosos autores, por lo que existe abundante literatura sobre algorítmos heurísticos para encontrar soluciones al problema de SI.

Debido a que las metaheurísticas son métodos de búsqueda estocástica de propósito general, han sido adaptadas en numerosas ocaciones para encontrar soluciones a problemas de optimización combinatoria. Las metaheurísticas proveen un esquema de búsqueda flexible, que permite recorrer el espacio de soluciones de diversos problemas. Aunque no garantizan optimalidad, en general, las metaheurísticas son capaces de encontrar buenas soluciones en espacios de búsqueda particularmente grandes. Por esta razón, durante los últimos años, muchos autores han estudiado el uso de metaheurísticas para encontrar soluciones al problema de SI.

En este sentido, algunos de los primeros trabajos se centran en adaptar metaheurísticas de trayectoria para conseguir soluciones al problema de SI. Los trabajos de \emph{Cerverón et al.} \cite{cerveron2001another} y \emph{Zhang et al.} \cite{zhang2002optimal} introducen modificaciones al algoritmo de \emph{Búsqueda Tabú}. No obstante, la mayoria de los estudios se han enfocado en adaptaciones de \emph{Algoritmos Evolutivos}. En la literatura se encuentran adaptaciones de \emph{Algoritmos de Estimación de Distribución} \cite{sierra2001prototype}, \emph{Algoritmo Genético Inteligente} \cite{ho2002design}, \emph{Algoritmo Memético Estacionario} \cite{garcia2008memetic}, etc. Destaca el trabajo de \emph{Cano et al.} \cite{cano2003using}, que realiza un estudio comparativo entre algoritmos heurísticos al problema de SI y adaptaciones de \emph{Algoritmo Genético Generacional} (GGA), \emph{Algoritmo Genético Estacionario} (SGA), \emph{CHC Adaptive Search Algorithm} (CHC) y \emph{Population-Based Incremental Learning} (PBIL).

Sin embargo, el principal obstáculo para la aplicación de metaheurísticas en el problema de SI, es que se requiere un alto número de iteraciones para encontrar soluciones que cumplan satisfactoriamente los objetivos del problema en términos de reducción de los datos y precisión de representación.

En el presente trabajo se realiza un estudio comparativo entre 5 metaheurísticas: GGA, SGA, CHC, PBIL y PSO. En función de los obstáculos planteados, se proponen una serie de modificaciones a las estrategias de generación de soluciones iniciales de estas metaheurísticas. La idea es iniciar el proceso de búsqueda en regiones del espacio de soluciones que mejoren los resultados en términos de los objetivos del problema. Esto con la finalidad de reducir el número de iteraciones necesarias para alcanzar buenas soluciones al problema y así permitir su aplicación sobre conjuntos de datos con gran número de instancias.

Este trabajo está estructurado como sigue. En el Capítulo 1 se describe en amplitud el problema de SI y los enfoques seguidos en la literatura para su solución. En el Capítulo 2 se describen de forma general las metaheurísticas usadas en este estudio, mientras que en el Capítulo 3 se presentan las modificaciones propuestas sobre las metaheurísticas para conseguir soluciones al problema de SI y resolver el obstáculo mencionado. En el Capítulo 4 se detalla el diseño experimental seguido y se presentan y analizan los resultados obtenidos. Finalmente, se presentan las conclusiones y recomendaciones finales del trabajo.
