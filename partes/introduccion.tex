\chapter*{Introducción}
\label{intro}
\lhead{\emph{Introducción}}
\addcontentsline{toc}{chapter}{Introducción}

{\color{red}

El avance de la ciencia y la tecnología durante las últimas décadas ha traido como consecuencia un aumento sin precedentes en la candidad de datos generados y recopilados por la actividad humana. El \emph{Proyecto Genoma Humano}, el \emph{Instituto SETI} y el \emph{Gran Colisionador de Hadrones}, tienen algo en común: generan una enorme cantidad de datos, por lo que resulta imposible usarlos y mucho menos analizarlos de forma tradicional.

Por esta razón, nuevos cambos de estudio, como el Descubrimiento de Conocimiento en Bases de Datos (\emph{KDD}) y Minería de Datos (\emph{DM}), emergen para afrontar el creciente problema que se genera al intentar usar y analizar enormes cantidades de datos.

Bajar complejidad, disminuir los datos.

}
