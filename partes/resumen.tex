\addtotoc{Resumen}
\abstract{
\addtocontents{toc}{\vspace{1em}}

En este trabajo se considera el problema de \emph{Selección de Instancias} (SI), como estrategia de reducción de datos durante la aplicación de procesos de KDD.
Estando los datos agrupados en instancias independientes (una entrada en una base de datos), el problema de SI busca seleccionar el subconjunto de instancias de menor cardinalidad, que mantenga o mejore la precisión de clasificación al ser usado como conjunto de entrenamiento.
El estudio de este problema se ha popularizado a lo largo de los últimos años debido a la creciente necesidad de reducir los datos a ser almacenados y posteriormente analizados.
Dada su inherente complejidad, los trabajos sobre el problema de SI se han centrado en la formulación de algoritmos heurísticos que puedan generar (en tiempos aceptables) buenas soluciones al problema.
Durante la última década, numerosos autores han acudido al uso de metaheurísticas para encontrar soluciones al problema de SI, debido a su comprobada capacidad para encontrar buenas soluciones a gran variedad de problemas.
En el presente trabajo se implementan 5 metaheurísticas (GGA, SGA, CHC, PBIL y PSO) para conseguir soluciones al problema de SI. Adicionalmente, se proponen una serie de modificaciones sobre las estrategias de generación de soluciones iniciales, con la finalidad de reducir el tamaño de las soluciones encontradas y guiar el inicio de la búsqueda a soluciones que satisfagan los objetivos del problema. La evaluación experimental muestra que la disminución de la probabilidad de aparición de bits al $5\%$ permite una disminución sustancial de los tiempos de ejecución de las metaheurísticas, mejorando los resultados en términos de la reducción alcanzada, y sin afectar negativamente la precisión del clasificador. Adicionalmente, un estudio comparativo entre las metaheurísticas evaluadas, concluye que PBIL obtiene los mejores resultados en función de los objetivos del problema. Sin embargo, el comportamiento exhibido por CHC resulta interesante con miras en su aplicación a conjuntos de datos de mayor tamaño, puesto que es la metaheurística que se ve menos afectada por el número de instancias.

}

% Las palabras clave son generalmente los nombres de áreas de investigación a
% los cuales está asociado el trabajo. Generalmente son tres o cuatro.
\noindent \begin{small} \textbf{Palabras clave}: reducción de datos, selección de instancias, metaheurísticas.
\end{small}
