\chapter*{Conclusiones y Recomendaciones}
\label{conclusiones}
\lhead{\emph{Conclusiones y Recomendaciones}}
\addcontentsline{toc}{chapter}{Conclusiones y Recomendaciones}

En el presente trabajo fueron implementadas 5 metaheurísticas bio-inspiradas (\emph{i.e.} GGA, SGA, CHC, PBIL y PSO) para encontrar soluciones al problema de Selección de Instancias. Para esto, se propusieron (en función de los objetivos del problema) una serie de modificaciones a las estrategias de generación de soluciones iniciales al problema, usadas por las metaheurísticas como punto inicial del proceso de búsqueda. Adicionalmente, se realizó un estudio comparativo entre las metaheurísticas implementadas.%, para determinar aquellas metaheurísticas con mayor capacidad para encontrar soluciones al problema.

Los aportes de este trabajo se centran en las modificaciones realizadas sobre la generación de soluciones iniciales. En primer lugar, se propuso disminuir la probabilidad de aparición de bits $\delta$ (normalmente fijada en $50\%$), con la finalidad de reducir la cardinalidad de las soluciones iniciales generadas. Al usar una probabilidad igual al $5\%$, se lograron los objetivos planteados en términos del tamaño de las soluciones, además de una disminución considerable en el tiempo de ejecución; todo esto sin afectar de manera significativa en el error de clasificación.

Adicionalmente, se propuso modificar la probabilidad de aparición de cada bit de forma individual, aumentanto la probabilidad de aparición de las instancias seleccionadas por algoritmos heurísticos al problema de SI. En este sentido, se evaluó el impacto de usar probabilidades en función de las soluciones generadas por CNN, NEHS, Closest NE y Farthest NE, en contraste con el uso de una probabilidad constante. Contario a lo esperado, esta estrategia no resultó particularmente beneficiosa; únicamente la probabilidad en base a la selección de NEHS logró resultados comparables (y en algunos casos mejores) con el uso de una probabilidad constante.

Una vez determinadas las estrategias a seguir para la generación de solucines, se procedió a realizar un estudio comparativo entre las metaheurísticas implementadas, con la finalidad de determinar aquellas metaheurísticas que se comportan consistentemente mejor que las demás. De los resultados obtenidos, se concluye que PBIL encuentra las mejores soluciones en función de los objetivos del problema de SI. Sin embargo, el tiempo de ejecución de PBIL se ve muy afectado por el número de instancias en los conjuntos de datos. En este sentido, los resultados reportados por CHC lo convierten en una opción a considerar al momento de buscar soluciones al problema de SI sobre conjuntos de datos de tamaños mayores a los evaluados en este estudio.

En función del trabajo realizado, existen algunos aspectos relevantes a tomar en cuenta en futuros estudios sobre el problema de SI. A continuación se presentan algunas direcciones para expandir el presente estudio:

\begin{itemize}
\item Emplear una evaluación estratificada (como la descrita por \cite{cano2003using}) para encontrar soluciones a conjuntos de tamaño mayor que los usados en el presente trabajo.
\item Analizar el impacto de generar soluciones iniciales mediante algoritmos heurísticos, en el proceso de búsqueda llevado a cabo por metaheurísticas de trayectoria.
\item Evaluar adaptaciones alternativas de PSO para encontrar soluciones con representación binaria, con el objetivo de reducir el impacto que tiene el tamaño de los conjuntos de datos sobre su tiempo de ejecución.
\item Profundizar el estudio de NEHS (y algoritmos básados en ideas similares), para determinar su capacidad para conseguir soluciones al problema de SI.
\end{itemize}
